\documentclass[../TGMAFFIRO.tex]{subfiles}

\begin{document}
\section{Markets}
\begin{definition}[\textbf{Markets}]
	A market $X(t) = \{X_i(t)\}_{i=0}^n$ is an $\salgF^{(m)}_t$-adapted $n+1$ It\^o Process where
	\begin{equation}
		dX_0 = \rho(t,\omega) X_0(t) dt; \ X(0) = 1; \label{eq:safe_investment}
	\end{equation}
	
	and
	\begin{equation}
		dX_i = \mu_i(t,\omega) dt + \sum_{j=1}^{m}\sigma_{ij}(t,\omega)dW_j(t) \label{eq:risky_asset}.
	\end{equation}
\end{definition}

We regard (\ref{eq:safe_investment}) as the safe asset in the market due the lack of a diffusion term. With this in mind, $\rho$ represents the average rate of return for the safe investment\\

On the other hand, (\ref{eq:risky_asset}) represent a risky asset. In this context, $\mu_i$ represent the drift term or, in other words, the average rate of return for asset $i$, while $\sigma_{ij}$ represents the amount of variation for the asset $i$ (note that we are generally defining a risky asset as a sum of different \textit{uncertainty} factors). In effect, $\sigma_{ij}$ is the diffusion term of the stock.\\

For a more convenient notation, let $W(t) = \{W_i(t)\}_{i=1}^{m}$ be represented by a $m\times 1$ matrix and $\sigma_i$ by a $1\times m$ matrix of elements $\{\sigma{ij}\}_{j=1}^{m}$ then,
\begin{equation}
  dX_i = \mu_i(t,\omega) dt + \sigma_idW(t).
\end{equation}

\begin{definition}[\textbf{Normalized Markets}]
	The market $\market$ is said to be normalized if $X_0(0)= 1$
\end{definition}

We may define a normalized market transforming every entry of $\market$ as $\bar{X_i}(t) = X_0^{-1}(t)X_i(t)$. It follows that
\begin{equation}
  \bar{X}(t) = (1, \bar{X}_1(t), \ldots, \bar{X}_n(t))
\end{equation}

As pointed out by \aycite{oksendal}, normalizing the market ``corresponds to regarding the price $X_0(t)$ of the safe investment as the unit of price (the numeraire) and computing the other prices in therms of this unit.''

\begin{remark}
	Since $dX_0 = \rho X_0 dt$, we see that $X_0(t) = e^{\int_0^t\rho(s,\omega) ds}$. Let
	\begin{equation}
		\nu(t) := X^{-1}_0(t) = e^{-\int_0^t\rho(s,\omega) ds}.
	\end{equation}
This implies
\begin{align}
	d\bar X_i &= d\left(\nu(t)X_i(t)\right) \nonumber \\
			&= \nu(t)dX_i + X_id\nu(t) + d\nu(t)dX_i(t)\nonumber\\
			&= \nu(t) [\mu_i dt + \sigma_i dW(t)] + X_i[\nu(t))(-\rho dt)] + \nonumber\\
			&\phantom{{}=1} [\nu(t)(-\rho dt)\cdot(\mu_i dt + \sigma_i dW(t))]\nonumber\\
			&=\nu(t)[(\mu_i - \rho X_i)dt + \sigma_i dW(t)] \label{eq:discounted_market}.
\end{align}

Considering (\ref{eq:discounted_market}) as $d\XBar_i(t) = \nu(t)[(\mu_i dt + \sigma_i dW(t)) - \rho X_i(t)] = \nu(t)[dX_i(t) - \rho X_i(t)]$, we can represent the discounted market in the following matrix form:
\begin{equation}
  d\XBar(t) = \nu(t)[dX(t) - \rho X(t)]. \label{eq:discounted_market_matrix}
\end{equation}

\end{remark}

\begin{definition}[\textbf{Portfolio}]
	A portfolio $\theta(t)$ in the market $\market$ is an ($n+1$)-dimensional $(t,\omega)$-measurable and $\salgF_t^{(m)}$-adapted stochastic process
	\begin{equation}
		\theta(t,\omega) = \{\theta_{i}(t,\omega)\}_{i=0}^{n} \ \forall \ t\in[0, T].
	\end{equation}
\end{definition}

The value $\theta_i(t,\omega)$ represents the amount of units for the asset $X_i$ held at time $t$.

\begin{definition}[\textbf{The value process}]
	The value at time $t$ for the portfolio $\theta$ is defined as
	\begin{equation}
		V(t) = \innerprod{\theta(t)}{X(t)} = \theta(t) \cdot X(t)
	\end{equation}
\end{definition}
The value $\theta_i(t,\omega)$ represents the amount of units for the asset $X_i$ held at time $t$.

\begin{definition}[\textbf{Self-financing portfolio}]
	The portfolio $\theta(t)$ is said to be self-financing if
	\begin{equation}
	\int_0^T\{|\theta_0\rho(s)X_0(s) + \sum_{i=1}^{n}\theta_i(s)\mu(s)| + \sum_{j=1}^m\left[\sum_{i=1}^n\theta_i(s)\sigma_{ij}(s)\right]^2\} ds < \infty,
	\end{equation}
and 
\begin{equation}
  dV(t) = \innerprod{\theta(t)}{dX(t)} \iff V(t) = V(0)  + \int_0^t \theta(s) dX(s).
\end{equation}

The change in the value of the portfolio is given uniquely by the change in the value of the asset; there is no influx or outflow of money for every $t\in(0,T]$.
\end{definition}

\begin{remark}
	Let $\theta$ be self-financing in the market $\market$ and denote
	\begin{equation}
		\bar V^\theta(t) = \theta(t)\bar X(t) = \nu(t) V^\theta(t)
	\end{equation}
the normalized (or discounted) value process.

Then,
\begin{align}
	d\valueProcessNorm{t	} &= \nu(t) d\valueProcess{t} + d\nu(t) \valueProcess{t} + d\nu(t) d\valueProcess{t}\nonumber\\
	&= \nu(t)\theta(t)dX(t) + \valueProcess{t}\nu(t)(-\rho(t)dt)\nonumber\\
	&= \nu(t)\theta(t)\left[dX(t) - X(t)\rho(t)dt\right]\nonumber\\
	&= \nu(t)d\XBar(t).\label{eq:change_norm_value_process}
\end{align}

Where (\ref{eq:change_norm_value_process}) follows from (\ref{eq:discounted_market_matrix}). We conclude that $\theta$ is also self-financing in $\normarket$.
\end{remark}


\begin{definition}[\textbf{Admissible Portfolio}]
	A self-financing portfolio is said to be admissible if $\valueProcessNorm{t}$ is (a.s) lower bounded. i.e., there exists $L < \infty$ such that
	\begin{equation}
		\valueProcess{t, \omega}  \geq -L \ \text{for a.a. } (t,\omega) \in [0,T]\times \Omega.
	\end{equation}
\end{definition}

\begin{definition}[\textbf{An arbitrage}]
	An admissible portfolio is said to be an arbitrage if $V(0) = 0$ and
	\begin{itemize}
		\item $\valueProcess{T} \geq 0$ a.s.; and
		\item $\Pm\left(\valueProcess{T} > 0 \right) > 0$.
	\end{itemize}
\end{definition}

It turns out that having a market constraint to only admissible portfolios lead to a market with arbitrage (see \aycite{oksendal}). We are interested in markets where no arbitrage is possible. If so, what requirements other constraints are necessary for $\market$ such that no arbitrage is possible?

\begin{definition}
	A measure $\Qm \sim \Pm$ for which the normalized price process $\normarket$ is a (local) martingale w.r.t. $\Qm$ is called an equivalent local martingale measure (EMM).
\end{definition}

\begin{proposition}
	Suppose there exists an equivalent local martingale measure $\Qm$ for the market $\normarket$ then $\market$ is an arbitrage-free market.
\end{proposition}

\begin{proposition}\label{prop:qmartingale-market}
	Suppose there exists an $m$-dimensional process $u$ in the extended family for It\^o integrals, where $X(t, \omega) = \{X_i(t,\omega)_{i=1}^n\}$ and
	\begin{equation}
		\sigma(t,\omega)u(t,\omega) = \mu(t, \omega) - \rho(t, \omega)X(t,\omega).
	\end{equation}

Let us define the measure $\Qm$, and the process $\QBm{t}$ such that
\[
	d\Qm(\omega) = e^{-\int_0^Tu(t,\omega) dW(t) - \frac{1}{2}\int_0^Tu^2(t,\omega) dt}d\Pm,
\]

and
\[
	\QBm{t} := \int_0^tu(s,\omega) ds + W(t).
\]
Then,
\begin{enumerate}
	\item $\QBm{t}$ is a Brownian motion and a $\salgF_t^{(m)}$-martingale w.r.t. $\Qm$;
	\item The representation of the normalized market $\normarket$ is given by
	\begin{align*}
		d\XBar_0(t) &= 0 \\
		d\XBar_i(t) &= \nu(t)\sigma_i d\QBm{t}
	\end{align*}
\end{enumerate}
\end{proposition}

\begin{proof}
	The first part of \ref{prop:qmartingale-market} follows from the Girsanov theorem. For the second one consider $d\XBar_i$ and (\ref{eq:discounted_market}) then,
	\begin{align*}
		d\XBar_i &= d(\nu(t)X_i(t)) \\
		&= \nu(t)[(\mu_i - \rho X_i)dt + \sigma_i dW(t)]\\
		&= \nu(t)[(\mu_i - \rho X_i)dt + \sigma_i (d\QBm{t} - u_i(t)dt)]\\
		&= \nu(t)[(\mu_i - \rho X_i)dt + \sigma_i (d\QBm{t} - \frac{\mu_i - \rho X_i(t)}{\sigma_i})dt]\\
		&= \nu(t)\sigma_id\QBm{t}. 
	\end{align*}
\end{proof}

\begin{definition}[\textbf{Claims}]
\begin{enumerate}
	\item A contigent claim $T$ is a lower bounded $\salgF_t^{(m)}$-measurable random variable $C(\omega)$;
	\item A claim $C(\omega)$ is said to be attainable if there exists an admissible portfolio $\theta(t)$ and $z\in\RNums$ such that
	\[
		C(\omega) = V_z^{\theta}(T) := z + \int_0^T \theta(t)dX(t); \text{ and}
	\]
	% with z as our initial fortune we can find an admissible portfolio θ(t) which generates a value V z θ (T) at time T which a.s. equals F: V_z^\theta(T,\omega) = C(\omega) for a.a, \omega
	\item A market $\market$ is said to be complete if every bounded claim at time $T$ is attainable.
\end{enumerate}
\end{definition}

%TODO: we are considering a model with two components, namely, the risk free asset and the risky asset from which we will derive the price from.
% C is a twice co
\begin{theorem}[\textbf{Black-Scholes Equation}]
	Consider a payoff $g(\omega)$ at time $T$. The price of the payoff at time $0\leq t\leq T$ is given by $C(t,S_t)$, where $C$ is the solution to the partial differential equation
	\begin{equation}
		\partialwrt{C}{t} + \frac{1}{2}\sigma^2S^2\partialwrt[2]{C}{S} + rS\partialwrt{C}{S} - rC = 0,
	\end{equation}

with final boundary condition $C(T,S_T) = g(S_T)$.
\end{theorem}

\begin{proof}
	Let $C(t, S_t)$ the value of the payoff at time $t$ and $dS_t = S_t(\mu dt + \sigma dW_t)$ the dynamics of the stock. The derivative of $C$ is given by
	\begin{align}
		dC(t,S_t) &= \partialwrt{C}{t}{(t, S_t)} dt + \partialwrt{C}{s}{(t, S_t)} dS_t + \frac{1}{2}\partialwrt[2]{C}{s}{(t, S_t)} (dS_t)^2 \nonumber \\
			&= \partialwrt{C}{t}{(t, S_t)} dt + \partialwrt{C}{s}{(t, S_t)} (\mu dt + \sigma dW_t) + \frac{1}{2}\partialwrt[2]{C}{s}{(t, S_t)} (S_t\sigma)^2 dt \nonumber\\
			&= \left(\partialwrt{C}{t}{(t, S_t)} + \mu S_t\partialwrt{C}{s}(t, S_t) + (S_t\sigma)^2\frac{1}{2}\partialwrt[2]{C}{s}{(t, S_t)} \right) dt  + S_t\sigma\partialwrt{C}{s}{(t, S_t)} \nonumber
	\end{align}

\end{proof}
\end{document}