\documentclass[../TGMAFFIRO.tex]{subfiles}

\begin{document}
\section{Markets}
\begin{definition}[\textbf{Markets}]
	A market $X(t) = \{X_i(t)\}_{i=0}^n$ is an $\salgF^{(m)}_t$-adapted $n+1$ It\^o Process where
	\begin{equation}
		dX_0 = r(t,\omega) X_0(t) dt; \ X(0) = 1; \label{eq:safe_investment}
	\end{equation}
	
	and
	\begin{equation}
		dX_i = \mu_i(t,\omega) dt + \sum_{j=1}^{m}\sigma_{ij}(t,\omega)dW_j(t) \label{eq:risky_asset}.
	\end{equation}
\end{definition}

We regard (\ref{eq:safe_investment}) as the safe asset in the market due the lack of a diffusion term. With this in mind, $r$ represents the average rate of return for the safe investment\\

On the other hand, (\ref{eq:risky_asset}) represent a risky asset. In this context, $\mu_i$ represent the drift term or, in other words, the average rate of return for asset $i$, while $\sigma_{ij}$ represents the amount of variation for the asset $i$ (note that we are generally defining a risky asset as a sum of different \textit{uncertainty} factors). In effect, $\sigma_{ij}$ is the diffusion term of the stock.\\

For a more convenient notation, let $W(t) = \{W_i(t)\}_{i=1}^{m}$ be represented by a $m\times 1$ matrix and $\sigma_i$ by a $1\times m$ matrix of elements $\{\sigma{ij}\}_{j=1}^{m}$ then,
\begin{equation}
  dX_i = \mu_i(t,\omega) dt + \sigma_idW(t).
\end{equation}

\begin{definition}[\textbf{Normalized Markets}]
	The market $\market$ is said to be normalized if $X_0(0)= 1$
\end{definition}

We may define a normalized market transforming every entry of $\market$ as $\bar{X_i}(t) = X_0^{-1}(t)X_i(t)$. It follows that
\begin{equation}
  \bar{X}(t) = (1, \bar{X}_1(t), \ldots, \bar{X}_n(t))
\end{equation}

As pointed out by \aycite{oksendal}, normalizing the market ``corresponds to regarding the price $X_0(t)$ of the safe investment as the unit of price (the numeraire) and computing the other prices in therms of this unit.''

\begin{remark}
	Since $dX_0 = r X_0 dt$, we see that $X_0(t) = e^{\int_0^tr(s,\omega) ds}$. Let
	\begin{equation}
		\nu(t) := X^{-1}_0(t) = e^{-\int_0^tr(s,\omega) ds}.
	\end{equation}
This implies
\begin{align}
	d\bar X_i &= d\left(\nu(t)X_i(t)\right) \nonumber \\
			&= \nu(t)dX_i + X_id\nu(t) + d\nu(t)dX_i(t)\nonumber\\
			&= \nu(t) [\mu_i dt + \sigma_i dW(t)] + X_i[\nu(t))(-r dt)] + \nonumber\\
			&\phantom{{}=1} [\nu(t)(-r dt)\cdot(\mu_i dt + \sigma_i dW(t))]\nonumber\\
			&=\nu(t)[(\mu_i - r X_i)dt + \sigma_i dW(t)] \label{eq:discounted_market}.
\end{align}

Considering (\ref{eq:discounted_market}) as $d\XBar_i(t) = \nu(t)[(\mu_i dt + \sigma_i dW(t)) - r X_i(t)] = \nu(t)[dX_i(t) - r X_i(t)]$, we can represent the discounted market in the following matrix form:
\begin{equation}
  d\XBar(t) = \nu(t)[dX(t) - r X(t)]. \label{eq:discounted_market_matrix}
\end{equation}

\end{remark}

\begin{definition}[\textbf{Portfolio}]
	A portfolio $\theta(t)$ in the market $\market$ is an ($n+1$)-dimensional $(t,\omega)$-measurable and $\salgF_t^{(m)}$-adapted stochastic process
	\begin{equation}
		\theta(t,\omega) = \{\theta_{i}(t,\omega)\}_{i=0}^{n} \ \forall \ t\in[0, T].
	\end{equation}
\end{definition}

The value $\theta_i(t,\omega)$ represents the amount of units for the asset $X_i$ held at time $t$.

\begin{definition}[\textbf{The value process}]
	The value at time $t$ for the portfolio $\theta$ is defined as
	\begin{equation}
		V(t) = \innerprod{\theta(t)}{X(t)} = \theta(t) \cdot X(t)
	\end{equation}
\end{definition}
The value $\theta_i(t,\omega)$ represents the amount of units for the asset $X_i$ held at time $t$.

\begin{definition}[\textbf{Self-financing portfolio}]\label{def:self-financing-portfolio}
	The portfolio $\theta(t)$ is said to be self-financing if
	\begin{equation}
	\int_0^T\{|\theta_0r(s)X_0(s) + \sum_{i=1}^{n}\theta_i(s)\mu(s)| + \sum_{j=1}^m\left[\sum_{i=1}^n\theta_i(s)\sigma_{ij}(s)\right]^2\} ds < \infty,
	\end{equation}
and 
\begin{equation}
  dV(t) = \innerprod{\theta(t)}{dX(t)} \iff V(t) = V(0)  + \int_0^t \theta(s) dX(s).
\end{equation}

The change in the value of the portfolio is given uniquely by the change in the value of the asset; there is no influx or outflow of money for every $t\in(0,T]$.
\end{definition}

\begin{remark}
	Let $\theta$ be self-financing in the market $\market$ and denote
	\begin{equation}
		\valueProcessNorm{t} = \theta(t)\bar X(t) = \nu(t) \valueProcess{t}
	\end{equation}
	
the normalized (or discounted) value process.

Then,
\begin{align}
	d\valueProcessNorm{t	} &= \nu(t) d\valueProcess{t} + d\nu(t) \valueProcess{t} + d\nu(t) d\valueProcess{t}\nonumber\\
	&= \nu(t)\theta(t)dX(t) + \valueProcess{t}\nu(t)(-r(t)dt)\nonumber\\
	&= \theta(t)\nu(t)\left[dX(t) - X(t)r(t)dt\right]\nonumber\\
	&= \theta(t)d\XBar(t).\label{eq:change_norm_value_process}
\end{align}

Where (\ref{eq:change_norm_value_process}) follows from (\ref{eq:discounted_market_matrix}). We conclude that $\theta$ is also self-financing in $\normarket$.
\end{remark}


\begin{definition}[\textbf{Admissible Portfolio}]
	A self-financing portfolio is said to be admissible if $\valueProcessNorm{t}$ is (a.s) lower bounded. i.e., there exists $L < \infty$ such that
	\begin{equation}
		\valueProcess{t, \omega}  \geq -L \ \text{for a.a. } (t,\omega) \in [0,T]\times \Omega.
	\end{equation}
\end{definition}

\begin{definition}[\textbf{An arbitrage}]
	An admissible portfolio is said to be an arbitrage if $V(0) = 0$ and
	\begin{itemize}
		\item $\valueProcess{T} \geq 0$ a.s.; and
		\item $\Pm\left(\valueProcess{T} > 0 \right) > 0$.
	\end{itemize}
\end{definition}

It turns out that having a market constraint to only admissible portfolios lead to a market with arbitrage (see \aycite{oksendal}). We are interested in markets where no arbitrage is possible. If so, what requirements other constraints are necessary for $\market$ such that no arbitrage is possible?

\begin{definition}
	A measure $\Qm \sim \Pm$ for which the normalized price process $\normarket$ is a (local) martingale w.r.t. $\Qm$ is called an equivalent local martingale measure (EMM).
\end{definition}

\begin{proposition}
	Suppose there exists an equivalent local martingale measure $\Qm$ for the market $\normarket$ then $\market$ is an arbitrage-free market.
\end{proposition}

\begin{proposition}\label{prop:qmartingale-market}
	Suppose there exists an $m$-dimensional process $u$ in the extended family for It\^o integrals, where $X(t, \omega) = \{X_i(t,\omega)\}_{i=1}^n$ and
	\begin{equation}
		\sigma(t,\omega)u(t,\omega) = \mu(t, \omega) - r(t, \omega)X(t,\omega).
	\end{equation}

Let us define the measure $\Qm$, and the process $\QBm{t}$ such that
\[
	d\Qm(\omega) = e^{-\int_0^Tu(t,\omega) dW(t) - \frac{1}{2}\int_0^Tu^2(t,\omega) dt}d\Pm,
\]

and
\[
	\QBm{t} := \int_0^tu(s,\omega) ds + W_t.
\]
Then,
\begin{enumerate}
	\item $\QBm{t}$ is a Brownian motion and a $\salgF_t^{(m)}$-martingale w.r.t. $\Qm$;
	\item The representation of the normalized market $\normarket$ is given by
	\begin{align*}
		d\XBar_0(t) &= 0 \\
		d\XBar_i(t) &= \nu(t)\sigma_i d\QBm{t}
	\end{align*}
\end{enumerate}
\end{proposition}

\begin{proof}
	The first part of \ref{prop:qmartingale-market} follows from the Girsanov theorem. For the second one consider $d\XBar_i$ and (\ref{eq:discounted_market}) then,
	\begin{align*}
		d\XBar_i &= d(\nu(t)X_i(t)) \\
		&= \nu(t)[(\mu_i - r X_i)dt + \sigma_i dW(t)]\\
		&= \nu(t)[(\mu_i - r X_i)dt + \sigma_i (d\QBm{t} - u_i(t)dt)]\\
		&= \nu(t)[(\mu_i - r X_i)dt + \sigma_i (d\QBm{t} - \frac{\mu_i - r X_i(t)}{\sigma_i})dt]\\
		&= \nu(t)\sigma_id\QBm{t}.
	\end{align*}
\end{proof}

\begin{definition}[\textbf{Claims}]
\begin{enumerate}
	\item A contigent claim $T$ is a lower bounded $\salgF_t^{(m)}$-measurable random variable $C(\omega)$;
	\item A claim $C(\omega)$ is said to be attainable if there exists an admissible portfolio $\theta(t)$ and $z\in\RNums$ such that
	\[
		C(\omega) = V_z^{\theta}(T) := z + \int_0^T \theta(t)dX(t) \text{a.s. and,}
	\]
	
	\begin{equation}
		\valueProcessNorm{T}	 = z + \int_0^t \nu(s)\sum_{i=1}^{n}\theta_i(s)\sigma_i(s) dW^\Qm_s \ \forall \ 0 \leq t \leq T
	\end{equation} 
	is a $\Qm$-martingale. If such $\theta$ exists, it is called a replicating portfolio.
	% with z as our initial fortune we can find an admissible portfolio θ(t) which generates a value V z θ (T) at time T which a.s. equals F: V_z^\theta(T,\omega) = C(\omega) for a.a, \omega
	\item A market $\market$ is said to be complete if every bounded claim at time $T$ is attainable.
\end{enumerate}
\end{definition}

\textbf{
In what follows, we will assume $u(t,\omega) \in \Nhatfam{0}{T}$ such that
\begin{equation}\label{eq:condition_change_measure}
	\sigma(t,\omega) u(t,\omega) = \mu(t,\omega) - r(t,\omega) X_t,
\end{equation}
and let $\Qm$, and $W_t^\Qm$ such that
\begin{align}
  d\Qm = e^{-\left(\int_0^t u(t,\omega) dW_t + \frac{1}{2}\int_0^T u^2(t,\omega) dt\right)} d\Pm \label{eq:emmq} \\
  W^\Qm_t := \int_0^t (s,\omega) ds + W_t
\end{align}
}



\begin{theorem}\label{th:arbitrage_free_market}
	Suppose there exists a process $u(t,\omega) \in \Nhatfam{0}{T}$ for $X(t,\omega) = (X_1(t,\omega), \ldots, X_n(t,\omega))$ such that
	\begin{equation}
		\sigma(t,\omega) u(t,\omega) = \mu(t, \omega) - \rho(t,\omega) X(t,\omega)
	\end{equation}
	for a.a. $(t,\omega)$ and such that
	
	\begin{equation}\label{eq:condition_arbitrage_free_market}
		\Exp{e^{\frac{1}{2}u^2(t,\omega) dt}} < \infty.
	\end{equation}
	
	Then the market has no arbitrage.
\end{theorem}

\begin{theorem}\label{th:complete_market}
	The market $\market$ is a compete if and only if there exists $\sigma(t,\omega)$ invertible for a.a. $(t,\omega)$:
		
	\begin{equation}
		\sigma^{-1}(t,\omega) \sigma(t,\omega) = I_m \Rightarrow Rank(\sigma) = m
	\end{equation}
\end{theorem}

\begin{theorem}\label{th:european_price}
	Let $C$ a European claim at time $T$ such that $\ExpMeasure{\Qm}{\nu(t)C} < \infty$ for a given complete market $\market$. Then, the price of $C$ is
	
	\begin{equation}
		p(C) = q(C) = \ExpMeasure{\Qm}{\nu(t)C}.
	\end{equation}

\end{theorem}

\section{Black-Scholes PDE}
For now on, we assume a market that consists of two securities: a bond (or cash account) $B_t$, and a stock $S_t$. Clearly, this market consists of a risk-free element, and a risky one. As before, the dynamics of these two assets are as follows:

\begin{align}
	dB_t &= r B_t dt  \label{eq:bond_dynamics} \\
	dS_t &= S_t(\mu dt + \sigma dW_t) \label{eq:stock_dynamics}
\end{align}

With $r, \mu, \sigma \in \RNums$.

Furthermore, for some claim $C$, we assume that this claim, at time $T$ has the form
\begin{equation}
  C(\omega) = g(T, S_T).
\end{equation}

With the tools we currently have, it is desirable to compute the actual value of $C$ at time $t=0$. The key idea behind the following models is that of replication. Intuitively put, the fair price anyone would be willing to pay for a contingent payoff (i.e., a derivative), is the cost that the seller of the option incurs at time $t=0$ in order to make a self-financing portfolio\footnote{That is, that exists a portfolio $\theta$ that satisfies definition \ref{def:self-financing-portfolio}.}.\\

In their paper ``The Pricing of Options and Corporate Liabilities'', \aycite{black-scholes} provide a replicating argument as to the price of the derivative. We state their result in the following theorem:

\begin{theorem}[\textbf{Black-Scholes Equation}]
	Consider a European claim $g(\omega)$ at time $T$. The price of the claim at time $0\leq t\leq T$ is given by $C(t,S_t)$, where $C$ is the solution to the partial differential equation
	
	\begin{equation}\label{eq:black-scholes-formula}
		\partialwrt{C}{t} + \frac{1}{2}\sigma^2S^2\partialwrt[2]{C}{S} + rS\partialwrt{C}{S} - rC = 0,
	\end{equation}

with final boundary condition $C(T) = g(T, S_T)$.
\end{theorem}

% We will prove the Black-Scholes formula vía replication following a rather informal, yet enlightning, argument
\begin{proof}
	Let $C(t, S_t)$ the value of the payoff at time $t$ and $dS_t = S_t(\mu dt + \sigma dW_t)$ the dynamics of the stock. The derivative of $C$ is given by
	\begin{align}
		dC(t,S_t) &= \partialwrt{C}{t}{(t, S_t)} dt + \partialwrt{C}{s}{(t, S_t)} dS_t + \frac{1}{2}\partialwrt[2]{C}{s}{(t, S_t)} (dS_t)^2 \nonumber \\
			&= \partialwrt{C}{t}{(t, S_t)} dt + \partialwrt{C}{s}{(t, S_t)} (\mu dt + \sigma dW_t) + \frac{1}{2}\partialwrt[2]{C}{s}{(t, S_t)} (S_t\sigma)^2 dt \nonumber\\
			&= \left(\partialwrt{C}{t}{(t, S_t)} + \mu S_t\partialwrt{C}{s}(t, S_t) + (S_t\sigma)^2\frac{1}{2}\partialwrt[2]{C}{s}{(t, S_t)} \right) dt  + S_t\sigma\partialwrt{C}{s}{(t, S_t)} dW_t\nonumber
	\end{align}
	
	In order to relax the notation, we denote $C_x = \partialwrt{C}{x}{(t, S_t)}$, and $C_{xx} = \partialwrt[2]{C}{x}{(t, S_t)}$. For which we rewrite
	\[
		dC(t, S_t) = \left(C_t + \mu S_t C_s + \frac{1}{2}(S_t\sigma)^2C_{ss} \right) dt  + S_t\sigma C_s dW_t
	\]
	
	Since $C(T, S_T)$ represents the contingent final payoff, then $C(0, S_0)$ is to be the price of the claim today. On the other hand, we know that the value of the replicating portfolio is given by
	
	\begin{equation} \label{eqproof:value_porfolio_bs_1}
		d\valueProcess{t} = \theta_0(t)dB_t  + \theta_1 dS_t
	\end{equation}
	
	Let us denote $\pi(t)$ as the total amount of money held in the stock at time $t$, i.e., value of the portfolio considering only the stocks. 	Recall that $\theta_0(t)$ and $\theta_1(t)$ represent total of units in the bank (resp. stock) held at time $t$. With this in mind, we can represent the total units held in stock as the total value given by the stocks divided by the market value of a single stock ($\pi(t) / S_t$). Conversely, we can represent the total units in the bank account as the value of the portfolio given by the value in the bank account divided by the value the bank account at time $t$ ($(\valueProcess{t} - \pi(t)) / B_t$) with this in mind, we may rewrite (\ref{eqproof:value_porfolio_bs_1}) as
	
	\begin{align}
		d\valueProcess{t} &=  \frac{\valueProcess{t} - \pi(t)}{B_t}dB_t + \frac{\pi(t)}{S_t} dS_t \nonumber\\
		&= \frac{\valueProcess{t} - \pi(t)}{B_t} r B_t dt + \frac{\pi(t)}{S_t} S_t(\mu dt + \sigma dW_t)\nonumber\\
		&= (\valueProcess{t} - \pi(t)) r dt  + \pi(t)(\mu dt + \sigma dW_t) 	\nonumber\\
		&= [r\valueProcess{t} + (\mu - r) \pi(t)] dt + \sigma\pi(t) dW_t
	\end{align}
	
	Where we note that the normalized process follows the dynamics:
	\begin{align}
		d\valueProcessNorm{t} &= d\left(\valueProcess{t} \nu(t)\right) \nonumber \\
		&= \nu(t) d\valueProcess{t} + d\nu(t) \valueProcess{t} + d\nu(t) d\valueProcess{t} \nonumber \\
		&= \nu(t)([r\valueProcess{t} + (\mu - r) \pi(t)] dt + \sigma\pi(t) dW_t) - r \valueProcess{t}\nu(t)dt\nonumber\\
		&= \nu(t)(\mu - r) \pi(t) dt + \nu(t)\pi(t) \sigma dW_t \nonumber \\
		&= \hat \pi(t) (\mu - r) + \hat \pi(t) \sigma dW_t
	\end{align}
	
	Having a replicating portfolio $V^\theta$, a claim $C$ to honor at time $T$, we need	
	\begin{equation}
		\valueProcess{T} = g(S_T) = C(T, S_T),
	\end{equation}
	i.e., if replication were to occur, the value of the portfolio at maturity must be the same as that of the payoff which, additionally, must be the value of the derivative at that time. It follows that
	
	\begin{equation}\label{eq:replication_condition}
		d\valueProcess{t} = dC(t, S_t).
	\end{equation}
	
	To compare (\ref{eq:replication_condition}) is to compare the stochastic element $dW_t$ and the deterministic component $dt$. We first observe that
	
	%TODO: Argue that this allows the perfect replication of the portfolio regarding risky assets.
	\begin{equation}
		\pi(t) \sigma dW_t = \sigma S_t \partialwrt{C}{s}{(t, S_t)}dW_t \iff \pi(t) = S_t C_s.
	\end{equation}
	
	Intuitively, the amount money needed of stock at time $t$ is given by the value of the stock $S_t$, and the change in the value of the stock at time $t$ w.r.t. the change in the underlying. In turn, this allows perfect replication of the claim at $t$. It then follows that
	
	\begin{align}
		&[r\valueProcess{t} + (\mu - r) \pi(t)] dt = \left(C_t + \mu S_t C_s + \frac{1}{2}(S_t\sigma)^2C_{ss} \right) dt\nonumber\\
		\Rightarrow & [r(\valueProcess{t} - \pi(t)] + \mu\pi(t) = \left(C_t + \mu S_t C_s + \frac{1}{2}(S_t\sigma)^2C_{ss} \right) \nonumber\\
		\Rightarrow &[r(C - S_t C_s] + \mu S_t C_s = C_t + \mu S_t C_s + \frac{1}{2}(S_t\sigma)^2C_{ss} \nonumber \\
		\Rightarrow &[r(C - S_t C_s] = C_t + \frac{1}{2}(S_t\sigma)^2C_{ss} \nonumber \\
		\Rightarrow & C_t + \frac{1}{2}(S_t\sigma)^2C_{ss} + rS_t C_s - rC = 0
	\end{align}
 \end{proof}
 
 \begin{remark}
 	The Black-Scholes formula (\ref{eq:black-scholes-formula}), together with condition (\ref{eq:replication_condition}) means that we can replicate the payoff $g(S_T)$ if we adjust the portfolio accordingly. Furthermore, the price of the derivative at time $t$ does not depend on the mean return rate of the underlying $\mu$. Solving for $C$ then, implies getting the price of the derivative for which replication can be performed.
 \end{remark}
 
 \section{Risk-Neutral Pricing}
  Pricing under replication denotes the key idea behind the Black-Scholes model: the price of the option today is the cost of the self-financing portfolio that I (the seller) should buy at the beginning of the contract in order to guarantee the final payoff.\\
 
  Consider once again the two-security market proposed, condition (\ref{eq:condition_change_measure}) becomes
\[
	\sigma S_t u(t,\omega) = \mu S_t - r S_t \Rightarrow u = \sigma^{-1}(\mu - \sigma)
\]


As a consequence, we have an equivalent martingale measure $\Qm$ as defined on (\ref{eq:emmq}), hence no arbitrage by \ref{th:arbitrage_free_market}, and a complete market by \ref{th:arbitrage_free_market}. We conclude that, by \ref{th:european_price}, that the fair price for both buyer and seller is $c$:

\begin{equation}\label{eq:risk-neutral-expectation}
  c = \ExpMeasure{\Qm}{\nu(T)g(S_T)}
\end{equation}

To make use of (\ref{eq:risk-neutral-price}), we will now turn to price a European call option, presented first in chapter 1. In the following proposition, we derive the famous closed-valued formula for the price of a European call option.

\begin{proposition}
	Consider a claim $C(\omega) = g(T, S_T) = \callPayoff{S_T}$ for some $K > 0$, the price of the claim at $t=0$, under risk neutral probability is
	
	\begin{equation} \label{eq:risk-neutral-price}
		c = S_0 \Phi(d_+) - Ke^{-rT}\Phi(d_-)
	\end{equation}
	Where,
	$\Phi(\cdot)$ is the c.d.f. of a standard normal distribution and,
	\[
		d_\pm = \frac{1}{\sigma\sqrt{T}}\left(\log\frac{S_0}{K} + (r \pm \frac{1}{2}\sigma^2)
		T\right)
	\]
\end{proposition}

\begin{proof}
	By (\ref{eq:risk-neutral-price}), the value of the payoff today is given by
	\begin{align}
		\ExpMeasure{\Qm}{\nu(T)g(S_T)} &= \ExpMeasure{\Qm}{\nu(T)\callPayoff{S_T}} \nonumber \\
		&= \ExpMeasure{\Qm}{\nu(T)(S_T - K)\ind_{S_T > K}} \nonumber\\
		&=\ExpMeasure{\Qm}{\nu(T)S_T\ind_{S_T > K}} - \nu(T)K\ExpMeasure{\Qm}{\ind_{S_T > K}}  \nonumber\\
		&= I_1 - K\nu(T)I_2 \label{eqproof:option_price1}
	\end{align}
	
	We now turn to rewrite the constraint of a call European option, namely $S_T > K$. Note that, under the $\Qm$ measure,
	\begin{align*}
		&S_T = S_0e^{\sigma\QBm{T} + (r - \frac{1}{2}\sigma^2)T}\\
		\iff &\log S_T = \log S_0 + \sigma\QBm{T} + (r - \frac{1}{2}\sigma^2)T
	\end{align*}
	Consider $S_T > K$ then,
	\begin{align*}
		&\log S_T > \log K\\
		\iff & \log S_0 + \sigma\QBm{T} + (r - \frac{1}{2}\sigma^2)T > \log K \\
		\iff & \QBm{T}  > \frac{1}{\sigma}\left(\log K  - \log S_0 + \frac{1}{2}\sigma^2T - rT\right).
	\end{align*}
	We now denote $\QBm{T} \equiv Z\sqrt T \sim N(0, \sqrt{T})$, where $Z\sim N(0,1)$. And conclude that
	\begin{equation}
		Z > \frac{1}{\sigma\sqrt T}\left(\log K  - \log S_0 + \left(-r + \frac{1}{2}\sigma^2\right)T\right) =: -d_-.
	\end{equation}
	
	Therefore, $S_T > K \Rightarrow Z > -d_-$.\\
	
	To get (\ref{eqproof:option_price1}), we start by solving $I_2$:
	\begin{align}
		\ExpMeasure{\Qm}{\ind_{S_T > K}} &= 1 \cdot \Qm(S_T > K) + 0 \cdot \Qm(S_T \leq K) \nonumber\\
		&= \Qm(S_T > K)\nonumber\\
		&= \Qm(Z > -d_-)\nonumber \\
		&= \Qm(Z < d_-)\nonumber \\
		&= \Phi(d_-)
	\end{align}
	
	Continuing with $I_1$, recall that $\nu(t) = e^{-rt}$. Then,
	\begin{align}
		\ExpMeasure{\Qm}{\nu(T)S_T\ind_{S_T > K}} &= e^{-rT}\ExpMeasure{\Qm}{S_t\ind_{S_T > K}} \nonumber \\
		&= e^{-rT}\ExpMeasure{\Qm}{S_0e^{\sigma\QBm{T} + (r - \frac{1}{2}\sigma^2)T}\ind_{S_T > K}} \nonumber \\
		&= S_0 \ExpMeasure{\Qm}{e^{\sigma\QBm{T} - \frac{1}{2}T\sigma^2}\ind_{S_T > K}} \nonumber \\
		&= S_0 \ExpMeasure{\Qm}{e^{\sigma\sqrt{T}z - \frac{1}{2}T\sigma^2}\ind_{Z > -d_-}} \nonumber \\
		&= S_0\int_{-d_-}^\infty e^{\sigma\sqrt{T}z - \frac{1}{2}T\sigma^2} \phi(z) dz \nonumber \\
		&= S_0\int_{-d_-}^\infty e^{\sigma\sqrt{T}z - \frac{1}{2}T\sigma^2} \frac{1}{\sqrt{2\pi}} e^{-\frac{1}{2}z^2} dz \nonumber \\
		&= \frac{S_0}{\sqrt{2\pi}}\int_{-d_2}^{\infty}e^{-\frac{1}{2}\left(z - \sigma\sqrt{T}\right)^2} dz.\label{eqproof:option_price2}
	\end{align}
	
	Let $u = z - \sigma\sqrt{T}$, and $d_+ := d_- + \sigma\sqrt{T}$. Then (\ref{eqproof:option_price2}) becomes
	\begin{align}
		&=S_0\left(\frac{1}{\sqrt{2\pi}}\int_{-d_+}^{\infty} e^{-\frac{1}{2}u^2} du\right) \nonumber \\
		&= S_0\Qm(Z > -d_+) \nonumber\\
		&= S_0\Qm(Z < d_+) \nonumber \\
		&= S_0 \Phi(d_+).
	\end{align}
\end{proof}

One may wonder about the relationship between risk-neutral pricing and the Black-Scholes PDE. Both models commit to the idea of hedging (or replication) of the payoff by removing the risk factor, i.e., the rate of return of the risky asset $\mu$ bears no importance in either model.\\

We may then pose the question: what is the value of $C$ for (\ref{eq:black-scholes-formula}) and (\ref{eq:risk-neutral-expectation})? Since we are talking about a complete and arbitrage-free market, we may assume that both paths yield the same result and, in fact, this is the case. By theorem \ref{th:feynman-kac} (the Feynman-Kac formula) with $f(x,t) = 0$, and $V(x,t) = r$ we see that

\begin{equation}
	\partialwrt{C}{t} + \frac{1}{2}\sigma^2S^2\partialwrt[2]{C}{S} + rS\partialwrt{C}{S} - rC = 0 \Rightarrow \ExpMeasure{\Qm}{\nu(T)g(S_T)}.
\end{equation}

\section{Montecarlo Valuation}
For a much more complex claims $C$ under the Black-Scholes model there may not exist a closed form formula such as for the European call option. In such cases, we can rely on (\ref{eq:risk-neutral-expectation}) and the law of large numbers to converge to the expected value (under the measure $\Qm$). Since we know the dynamics of the underlying under the $\Qm$ measure, namely $dS_t = S_t\left(rdt + \sigma dW_t\right)$, it is possible simulate the path of the process up to time $T$ and compute the mean payoff at maturity.\\

In this work we will not present a formal approach to this method, but rather provide an empirical analysis on the use cases for this method.\\

\begin{figure}[h!]
  \includegraphics[width=0.90\textwidth]{../images/european_call_montecarlo}
  \label{fig:montecarlo_simulation}
  \caption{250 simulation of a European Call under $\Qm$.}
\end{figure}


As an example, consider a European call option with $S_0 = 100$, $K = 92$, $r = 0.06$, $\sigma = 0.23$, and $T = 150 / 365$. If, at $t=T$, the value of the underlying falls below the value of the strike, then a payoff happens at time $T$. Figure \ref{fig:montecarlo_simulation} represents 250 simulated paths under the specified conditions; any simulated path below $K$ at maturity is highlighted.\\

By (\ref{eq:risk-neutral-price}), the price of the claim today with the parameters specified above is $c\approx 12.1253$. In order to simulate and compute the expected value of the claim at maturity, we only need to simulate the value of the underlying at maturity, that is, $S_T$. The general approach to the mean value of the claim $C$ at time $T$ can be represented as follows:\\

\begin{algorithm}[H]
	\SetAlgoLined
	\KwData{$S_0$, $K$, $r$, $\sigma$, $T$, $nrounds$, $nsim$}
	\KwResult{Mean price of the claim at $t=0$}
	Set $X \leftarrow \{0\}_{i=1}^{nrounds}$\;
	\For{$round \leftarrow 1$ \KwTo $nrounds$}{
	$totalPrice \leftarrow 0$\;
		\For{$sim \leftarrow 1$ \KwTo $nsim$}{
			$S_T = S_0e^{(r - \sigma^2/2)T + \sigma W_T}$\;
			$C = \max\{S_T - K, 0\}$\;
			$totalPrice \leftarrow totalPrice + C$
		}
		$X_{round} \leftarrow e^{-rT} \cdot \frac{totalPrice}{nsim}$\;
	}
	$c \leftarrow \frac{1}{nrounds} \sum_{j=1}^{nrounds}X_j$\;
	\Return $c$\;
	\caption{Montecarlo Pricing for a European Call Option}
\end{algorithm}

\hfill \break
Algorithm 1 is the simplest approach, and the highlights the main idea behind Montecarlo valuation. A more optimized approach is to vectorize the algorithm. For purposes of this work, we took 10,000 rounds with 100,000 simulations in between which yielded a mean price of $\hat c \approx 12.12566$, and a standard error $SE(c) \approx .04007$.

\begin{figure}[h]
	\centering
  \includegraphics[width=0.5\textwidth]{../images/montecarlo_distribution}
  \caption{Mean Ddstribution of a the present value of simulated call option.}
\end{figure}

\end{document}