\documentclass[../TGMAFFIRO.tex]{subfiles}

\begin{document}
In order to solve real-world problems, mathematical models display an approximation of reality. What makes a model exceptional is puzzling and ambiguous nonetheless, some, are able to skew the world's point of view towards a problem to the extend that they are used as a building block to bigger, more complex related problems.\\

The (in)famous Black-Scholes Formula, as it is commonly known, aimed to solve a real world problem, that of pricing a European call option. Its beauty, lies in the representation of a fundamentally convoluted and stochastic problem into a closed form solution. Students of mathematical finance encounter this formula whilst learning basic valuation theory which can be either presented as a deeply esoteric subject or an accepted law that outputs a correct value.\\

The aim of this work is to merge both views of the formula in a comprehensive approach for any newcomer to study. To do so, we gradually build the model, while explaining and aiming to show the relationship with the world it tries to represent.
\end{document}
