\documentclass[11pt]{report}

% Packages to use
\usepackage{amsmath}
\usepackage{amssymb}
\usepackage[T1]{fontenc} %uncomment to display chapters
\usepackage{titlesec, color}
\usepackage[numbers]{natbib}
\bibliographystyle{plainnat}
\usepackage{array}
\usepackage{csquotes}
\usepackage{url} % insert Urls in bib
\usepackage{tikz} % images
\usepackage{pgfplots} %images
\usepackage{standalone} % Load tikz from library 
\usepackage{mathrsfs} %Pretty fonts
\usepackage{enumerate} %Roman numerals
\usepackage{amsmath}
\usepackage{amssymb}
\usepackage{amsfonts}
\usepackage{amsthm} % For proofs



%% Defining mathematical environments
\newtheorem{definition}{Definition}[chapter]
\newtheorem{remark}{Remark}[chapter]
\newtheorem{proposition}{Proposition}[chapter]


%%%% COMMANDS %%%%
% sigma-algebra
\newcommand{\salg}{\sigma\text{-algebra}}
% borel sigma-algebra
\newcommand{\borelsalg}{\mathscr{B}(\mathbb{R})}

% Document properties
\title{Theoretical Grounds and Market Adaptations of Financial Fx and Interest Rate Options}
\author{Gerardo Dur\'an Mart\'in}

% Command: Chapter Style
\definecolor{gray75}{gray}{0.75}
\newcommand{\hsp}{\hspace{20pt}}
\titleformat{\chapter}[hang]{\Huge\bfseries}{\thechapter\hsp\textcolor{gray75}{|}\hsp}{0pt}{\Huge\bfseries}

% Cite authors. Use \citeauhor*{auth}

% Set line breaks inside cells
\newcommand{\breakcell}[2][c]{%
  \begin{tabular}[#1]{@{}c@{}}#2\end{tabular}}

% Cite as: AUTHOR(YEAR) style
\newcommand{\aycite}[1]{%
 \citeauthor{#1} (\citeyear{#1})}
 
% .............. It starts here ..............
\begin{document}
\tableofcontents

\chapter{Financial Markets}
Throughout modern history, the value of \textit{anything} has been of great importance. In order to set the value of exchange in a transaction, any two parties involved must agree in the current worth of whatever it is that will be transacted, and take opposite sides of interest in it\footnote{In other words, one party wants to buy (obtain) and the other to sell (get rid off)}. Furthermore, these two parties require to communicate, and be able to find one another or a third whose worth of the \textit{something} at play favors any one of them more.\\

A \textbf{security}, is the epitome of a tradable object in financial markets. We define a security as an instrument that either represents ownership or that derives its value from a commodity. We say that a security is \textbf{fungible} if any unit of the instrument is economically indistinguishable from all other units. That is, whomever buys (or sells) the instrument can pay (or receive) any unit at the same price at a given point in time.\\

The problem of searching for the price and the party whose view on the value of a security reflect ours is called \textbf{trading}. A \textbf{market} is the place where buyers meet sellers. Markets can be both a physical place (called a trading floor), or an electronic system.\\

Not all financial products are traded in formal markets. Over the counter (OTC) trades happen between two parties directly or via a broker, whom helps them settle the trade. 

\section{The Participants}
In a financial market, participants have various reasons to trade. On the one hand there is the \textbf{buy-side}. They trade in order to solve a financial problem originated outside of the market. This group sees the market as a means to an end, and thus, they rely on the resources available in the market.\\

\begin{table}[!h]
	\centering
	\begin{tabular}{ c c m{3cm} c}
		\hline
		Trader Type & Examples & Why They Trade & Instruments\\
		\hline
		Investor & \breakcell{Individual \\ Corporate pension funds \\ Insurance funds \\ Charitable and legal trusts \\ Endowments \\ Mutual Funds \\ Money Managers} & {To move wealth from the present to the future for themselves or for their clients} & \breakcell{Stocks \\ Bonds} \\
		\\
		Borrowers & \breakcell{Homeowners \\ Students \\ Corporations} & {To move wealth from the future to the present} & \breakcell{Mortgages \\ Bonds \\ Notes}\\
		\\
		Hedgers & \breakcell{Farmers \\ Manufacturers \\ Miners \\ Shippers \\ Financial Institutions} & {To reduce business operating risk} & \breakcell{Futures contracts \\ Forward contracts \\ Swaps}\\
		\\
		\breakcell{Asset \\ Exchangers} & \breakcell{International corporations \\ Manufacturers \\ Travelers} & {To acquire an asset that they value mire than the asset that they tender} & \breakcell{Currencies \\ Commodities} \\
		\\
		Gamblers & Individuals & {To entertain themselves} & Various\\
		\hline
	\end{tabular}
	\caption{Buy-side of the market. \aycite{harris}}
\end{table}

%
 On the other hand are the participants who offer exchange services to the buy-side. Their trading purpose is to satisfy the needs of the buy-side by taking the opposite side of the trade. This group sells exchange services to the buy-side, consequently they are known as the \textbf{sell-side}.\\

\begin{table}[h!]
    \centering
    \begin{tabular}{c c m{3cm}}
    	\hline
    	Trader Type & Examples & Why They Trade \\
    	\hline
    	Dealer & \breakcell{Market Maker \\ Specialist \\ Floor Trader \\ Locals \\ Day traders \\ Scalpers} & {To earn trading profits by supplying liquidity} \\
    	\\  	
    	Brokers & \breakcell{Retail brokers \\ Discount brokers \\ Full-service brokers \\ Institutional Brokers \\ Block brokers \\ Futures commission merchants} & {To earn commissions by arranging trades for clients} \\
    	\\
    	Broker-dealers & \breakcell{Wirehouses} & {To earn profits and trading commissions} \\
    	\hline
    \end{tabular}
    \caption{Sell-side of the market. \aycite{harris}}
\end{table}

We can classify the buy-side and the sell-side as those who require liquidity and as those who provide it. \textbf{Liquidity} is an important concept in financial markets. Although the term itself can be regarded in many ways, we will denote it as the likelihood of trading constraint to units available, price offered and time.

% TODO: Add present information about the current state of these products.
\section{The Instruments}
Financial instruments comprise a wide array of products. Each one of these serve a purpose to either the buy-side or the sell-side. We will review briefly the instruments that will serve for the development of this work.

\subsection{Bonds}

According to the Securities and Exchange Commission (SEC),
\begin{displayquote}
``A bond is a debt obligation, like an IOU. Investors who buy corporate bonds are lending money to the company issuing the bond. In return, the company makes a legal commitment to pay interest on the principal and, in most cases, to return the principal when the bond comes due, or matures.''
\end{displayquote}

A bond can be issued either by a corporation or by a government. Its main function is to raise capital for the issuer of the bond, known as the \textbf{debtor}, who promises a payment to the buyer of the bond, or as the \textbf{creditor} or the \textbf{bondholder}. The amount of money to be paid at the time of maturity is the \textbf{nominal value} or the \textbf{principal} of the bond.\\

In addition, bonds may or may not pay an interest on the nominal of the bond. In the former case, this is known as a \textbf{coupon bearing bond}, while the latter is referred to as the \textbf{zero coupon bond}.

\subsection{Stocks}
A stock is a security that represents ownership on a fraction of a corporation; it is a proportional division of a company's assets and distributed through what is known as a \textbf{dividend}. Future dividends are generally not known in advance. This contrast a big difference between bonds and stocks, while the former has a predefined number of payments, the latter is uncertain in amount and frequency.

\subsection{Foreign Exchange Currencies}
\aycite{kozy} defines a foreign currency as ``one country's currency freely convertible in the foreign exchange market''.\\

The foreign exchange market is an OTC market where buyers and sellers get together to buy and sell foreign currencies or financial contracts on said currencies. The foreign exchange market is also known as the \textit{FX market}.

\subsection{Derivatives}
Derivatives are the main topic in this work. \aycite{hull} defines derivatives as ``as a financial instrument whose value depends on (or derives from) the values of other, more basic, underlying variables''. This definition leaves room for many possible products that can depend on more than one factor.\\

Derivatives are mainly used to hedge and/or arbitrage. We can categorize them into two main categories: listed and OTC contracts. Listed contracts can be bought or sold in an organized market. These are standardized products that can be traded in financial markets. OTC contracts, on the other hand, represent a contract between two parties without the need of a market for the two parties to meet.

\section{Risk: Hedging and Arbitrage}
Risk is an inherent aspect of financial markets. For bonds, for example, risk is represented as the plausibility of default for the government or the company which issued the asset. For stocks, risk is presented as uncertainty of future prices, whether the company goes bankrupt, or the amount of the future dividend to be payed.\\

From a financial institution's perspective interested in selling derivative contracts, their intention may not be to speculate with the products they sell, as to where the price of the contracts may go, but rather to sell this a service. This forces the financial institution to eliminate all possible risk associated with the contract. Furthermore, the price they sell this contract for must be such that no one else may take a riskless profit from it.\\

Two important concepts emerge for a financial institution interested in covering the needs of the sell-side. That of \textbf{hedging} and \textbf{arbitrage}. Although we will define it more rigourously in upcoming chapters, the following will motivate the idea of what we want to achieve\\

 Consider the following example. Suppose we are bank willing to sell a stock three months from now. Since we do not know the price the stock will take three months from now and we are not interested in taking risk selling this \textit{derivative}, we wrongly decide to price this derivative as the current price of the stock plus an arbitrary fee, thus allowing us to deliver the stock three months from now and get a profit from it.\\
 
 One clever trader realizes that he can borrow, for three months, the current price of the stock, buy the stock, and pay the interested the bank imposes plus the nominal three months from now. If the value of the nominal plus interest is greater than the price for which the bank is selling this derivative, the trader can short-sell the stock, invest the amount received and enter the contract.\\
 
  Three months from now the trader must pay the price of the contract, which he can do since the amount he invested in the bank is greater than the price of the contract. Also, he can return the stock he shorted since the bank will give it to him, thus making a riskless profit.
  
\newpage

Although the bank decided cover its position in order to fulfill its future obligation, the price it gave the contract was such that the trader made a profit without taking any risk from it. Strictly speaking, the bank \textbf{hedged} its position and the trader made an \textbf{arbitrage}.\\

This simple example presents the key ideas to take account whenever we try to price any derivative. By replicating what the trader did, the bank would be hedged and would not incur in anyone making an arbitrage. Composing an effective hedge may not be so easy for every contract.\\

Consider a derivative that provides to the buyer the option to buy a stock at a future date and at a predetermined price. What we would like is to know the present value of this derivative, known as as the premium, such that it allow us for an effective hedge and prevent anyone from making an arbitrage.\\

At maturity date, the payoff for the option will be

\begin{equation}
	\max\{S-K, 0\}
\end{equation}

This contract is known as an European Call Option.

%% Importing Call payoff from images folder
\begin{figure}[h]
	\centering
	\includestandalone{images/Call}
	\caption{European Call Payoff}
\end{figure}

\newpage

The knowledge and tools we currently have can only take us so far as to know the price for this derivative. For us to effectively price this \textbf{option}, we will need tools to explain the random movements of stocks in pursuance of the desired premium.

\chapter{Probability Theory}

\section{Measurable Spaces}
\begin{definition}{\textbf{Algebra}}
	Let $\Omega$ be a set of points $\omega$. A system $\mathscr{A}$ of subsets of $\Omega$ is an Algebra if:
	\begin{enumerate}[I.]
		\item $\Omega \in \mathscr{A}$
		\item If $A, B \in \mathscr{A} \Rightarrow A \cap B \in \mathscr{A}$
		\item If $A \in \mathscr{A} \Rightarrow A^\mathsf{c} \in \mathscr{F}$
	\end{enumerate}
\end{definition}

\begin{definition}{\textbf{$\salg$}}\label{salgebra}
	A system $\mathscr{F}$ of subsets of $\Omega$ is a $\salg$ if it is an algebra and it satisfies one additional condition:
	\begin{equation*}
		\text{If } \{A_n\}_{n\geq 1} \in \mathscr{F} \Rightarrow \bigcup_{k=1}^{\infty} A_k \in \mathscr{F} 
	\end{equation*}
\end{definition}

\begin{remark}
	for $\{A_n\}_{n\geq 1} \in \mathscr{F}$ then, definition \ref{salgebra} is equivalent to say that
	\begin{equation*}
		\text{If } \{A_n\}_{n\geq 1} \in \mathscr{F} \Rightarrow \bigcap_{k=1}^{\infty} A_k \in \mathscr{F}
	\end{equation*}
\end{remark}

\begin{proof}
	By the generalized form of De Morgan's laws, $(\bigcap_k A_k)^\mathsf{c} \equiv \bigcup_k A_k^\mathsf{c}$. Since $A_k \in \mathscr{F} \ \forall \ k$ then, by definition,  $A_k^\mathsf{c} \in \mathscr{F} \ \forall \ k$ which implies  $\bigcup_k A_k^\mathsf{c}$ and this complement is also in $\mathscr{F}$
\end{proof}

\begin{definition}\label{salgebra_generated}
	Let $A \subseteq \Omega$ then, $\mathscr{F}_A = \{\emptyset, \Omega, A, \bar{A}\}$ is known as the $\salg$ generated by $A$. It is also denoted as $\sigma\{A\}$
\end{definition}

Defintion \ref{salgebra_generated} is, by definition, the smallest $\salg$ that cointains $A$. This definition is useful in the construction of one particular $\salg$ of subsets of $\mathbb{R}$.\\

Consider the collection of open intervals $(a,b) \in \mathbb{R} \ \forall \ a \leq b$. The smallest $\salg$ generated by this collection is known as the Borel $\salg$ of subsets of $\mathbb{R}$.

\begin{definition}{\textbf{The Borel $\salg$}}\label{borel_sigma_alg}
		\begin{equation}
			\borelsalg := \sigma\{(a,b) \subseteq \mathbb{R}:a\leq b\}
		\end{equation}
\end{definition}

Definition \ref{borel_sigma_alg} is the smallest $\salg$ that contains all open intervals in the real line. Furthermore, it can be proven that $[a, b]$, $(a,\infty)$, $(-\infty, b)$, $[a, b)$, $(a, b]$ are all elements in $\borelsalg$. 

\begin{definition}{\textbf{Measurable Space}}
	If $\mathscr{F}$ is a $\salg$ of subsets of $\Omega$, $(\Omega, \mathscr{F})$ is said to be a measurable space. If $A$ is a set in $\mathscr{F}$, we say that $A$ is $\mathscr{F}$-measurable (or measurable w.r.t. $\mathscr{F}$)
\end{definition}

\begin{definition}{\textbf{Measure}}
	Let $\bar{\mathbb R} := \mathbb R \cup \{-\infty, \infty\}$ be the extended real line and let $(\Omega, \mathscr{F})$ be a measurable space. It is said that a function $\mu: \mathscr{F} \rightarrow \bar{\mathbb R}$ is a measure over $(\Omega, \mathscr{F})$ if
	\begin{enumerate}[I.]
		\item $\mu(\emptyset) = 0$
		\item $\mu(A) \geq 0\ \forall \ A \in \mathscr{F} $
		\item $\mu$ is $\sigma$-additive: If $\{A_k\}_{k=1}^{\infty}$ is a set of disjoint events in $\mathscr{F}$ ($A_i \cap A_j = \emptyset \ \forall \ i \neq j$), 
		\begin{equation*}
			\bigcup_{k=1}^{\infty} \mu(A_k) = \mu(\sum_{k=1}^{\infty} A_k)
		\end{equation*}
	\end{enumerate}
\end{definition}


If this last definition holds true then, $(\Omega, \mathscr{F}, \mu)$ is said to be a \textbf{measure space}. $\mu(A)$ is the measure of $A\in\mathscr{F}$. Furthermore, if $\mu(\Omega) < \infty$ then, $\mu$ is said to be a finite measure. Lastly, if  $\mu(\Omega) = 1$ then, $\mu$ is said to be a probability.\\

If $\mu$ is a probability then, $\mu \equiv \mathbb{P}$ is said to be a probability space. We can formalize this definition as follows:

\begin{definition}{\textbf{Kolmogorov's Axiom System}}
	An ordered triple $(\Omega, \mathscr{F}, \mathbb{P})$ where
	\begin{itemize}
		\item $\Omega$ is a set of points $\omega$
		\item $\mathscr{F}$ is a $\salg$ of subsets of $\Omega$
		\item $\mathbb P$ is a probability on $\mathscr{F}$
	\end{itemize}
	is called a \textbf{probability space} or \textbf{model} (of an experiment)
\end{definition}

\section{Random Variables}
\begin{definition}{\textbf{Random Variable}}\label{rvar}
	Let $(\Omega, \mathscr{F}, \mu)$ be a measure space. The function $X: \Omega \rightarrow \mathbb{R}$ is an $\mathscr{F}$-measurable function if
	\begin{equation}
		X^{-1}(B) = \{\omega \in \Omega \ | \ X(w) \in B\} \in \mathscr{F} \ \ \forall \ \ B \subseteq \mathbb{R}
	\end{equation}
\end{definition}

\begin{definition}\label{dist_func}
	Consider the event $\{X \leq x\} := X^{-1}(-\infty, x]$. The function $F_X: \mathbb{R} \rightarrow [0, 1]$ defined as:
	\begin{equation}
		F_X(x) := \mathbb{P}\{X \leq x\}
	\end{equation}
	is known as the \textbf{distribution function} of the random variable $X$.
\end{definition}

Furthermore, a probability distribution function has the following properties:
\begin{itemize}
	\item For $x \leq y$ then, $F_X(x) \leq F_X(y)$
	\item $\lim\limits_{x\to\infty} F_X(x) = 1$ and $\lim\limits_{x\to - \infty} F_X(x) = 0$
	\item $F_X$ is right-continuous, meaning that $\lim\limits_{y\to x^+} F_X(y) = F_X(x)$ for every $x \in \mathbb{R}$
\end{itemize}

\begin{definition}
	A random variable $X$ is said to be \textbf{absolutely continuous} is there exists a nonnegative borel-measurable function $f: \mathbb{R}\rightarrow\mathbb{R}$ such that 
	\begin{equation}
		F_X(x) = \int_{-\infty}^{x} f(t) dt
	\end{equation}
\end{definition}

The function $f$ is known as the \textbf{probability density function} of the random variable $X$.

\section{Lebesgue Integrals and Expectations}

% Left to add:
%	- Lesbesgue Integral, Expectations and changes of measure, randon-nikodym
%	- Stochastic Processes: a collection of random variables index by a set space T

\chapter{Random Walks and Brownian Motion}

\section{Random Walks}
As stated in the first chapter, the behavior of stocks in financial markets behaves stochastically. We cannot be certain of the value that any stock will take some future time, no matter how small the future time-interval is.\\

Consider a more simple example of a stochastic process: Suppose that two players (A and B) toss an even coin $d$ times, one toss for every unit of time\footnote{Note that this unit of time can be one second, one day, 5 minutes, etc.}. Whenever tails land, A will pay B \$1. On the contrary, A will receive \$1 from B. Suppose further that we decide to model how much did A won (or lost) after $n$ tosses. To do so, we define the following:

\begin{definition}\label{srwalk}
	Let $\{X_j\}_{j\geq0}$ be a set of i.i.d. random variables such that $\mathbb{P}(X_i = \pm 1) = \frac{1}{2}$; $i \geq 1$. Define $M_0 = 0$ and
	\[M_n := \sum_{k=1}^n X_k \ \forall \ k \in \mathbb{N}\] 
	The process $\{M_d\}_{d\geq 0}$ is known as a \textbf{symmetric random walk}
\end{definition}

%% Importing symmetric random walk figure
\begin{figure}[h]
	\centering
	\includestandalone{images/symmetric_random_walk}
	\caption{Symmetric Random Walk}
\end{figure}

Definition \ref{srwalk} captures what we intended to model. From the outset of the game, both players start with with \$0 and, as the game progresses, whichever amount A wins, B looses it. These type of games, for which whatever one players wins, the other looses it are called \textbf{Zero-sum games}.\\

From here on we can start forming questions about the intrinsic behavior of the game. 

\begin{proposition}\label{expectation_srw}
	The expectation of a symmetric random walk is 0.
\end{proposition}

\begin{proof}
	Let $n \in \mathbb{N}$ then,
	\begin{align}
	\mathbb{E}(M_n) &= \mathbb{E}\left[X_0 + \sum_{i=1}^d X_i\right] \nonumber \\
				    &= \mathbb{E}[X_0] + \sum_{i=1}^d \mathbb{E}\left[X_i\right] \label{srw_jump}\\
				    &= 0 + n  \left( 1 \cdot \frac{1}{2} + (-1) \cdot \frac{1}{2}\right) \nonumber
	\end{align}
\end{proof}

Note That eq. \ref{srw_jump} is true because $X_0$ is a constant, and $X_i$, $i \geq 1$ are all i.i.d. random variables. Also, after a large number of times for which the game is played we expect to break even. Nobody would win or loose once we played a large number of times\footnote{In fact, according to Law of Large Number, we would expect to break even after a number so large that it approaches to infinity.}\\

This simple process can be transformed into a process that takes $n$ steps between any two units of time. This new process

\chapter{A Primer on Option Pricing}
\chapter{Stochastic Calculus}
\chapter{The Black-Scholes-Merton Formula}
\chapter{Pricing Under Real Market Conditions}
\newpage
\nocite{*}
\bibliography{ref}
\end{document}